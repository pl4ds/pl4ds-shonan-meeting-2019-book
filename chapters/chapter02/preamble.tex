\usepackage[utf8]{inputenc}
\usepackage{mathpartir}
%\usepackage{type1cm}        % activate if the above 3 fonts are
% not available on your system

\usepackage{makeidx}         % allows index generation
\usepackage{graphicx}        % standard LaTeX graphics tool
% when including figure files
\usepackage{multicol}        % used for the two-column index
\usepackage[bottom]{footmisc}% places footnotes at page bottom
\usepackage{url}
\usepackage{newtxtext}       %
\usepackage{newtxmath}       % selects Times Roman as basic font

% \usepackage{natbib}
\usepackage{footmisc}

%% Additional packages added. Add necessary packages here.
%\usepackage[english]{babel}
\usepackage{siunitx}
\usepackage{amssymb}
\usepackage{pifont}
\usepackage{xcolor}
\usepackage{tabularx}
\usepackage{listings}
\usepackage{booktabs}
\usepackage{hyperref}
\usepackage{url}
\usepackage{mathtools}
\usepackage{lipsum}
\usepackage{import}
\usepackage{bibunits}
\usepackage{acronym}
\usepackage[nottoc]{tocbibind}
\usepackage{numberpt}
\usepackage{stmaryrd}

\usepackage{listings}
\usepackage{upquote}
\usepackage{color}

\definecolor{bluekeywords}{rgb}{0.13,0.13,1}
\definecolor{greencomments}{rgb}{0,0.5,0}
\definecolor{turqusnumbers}{rgb}{0.17,0.57,0.69}
\definecolor{redstrings}{rgb}{0.5,0,0}

\lstdefinelanguage{scribble}{
  morekeywords={
  	global, protocol, role, from, to, interruptible, with, do, instantiates, par, and, rec, continue, choice, at, initiates, handle, returning, call, local, or, self
  },
  otherkeywords={ },
  keywordstyle=\color{bluekeywords},
  sensitive=true,
  %basicstyle=\scriptsize\ttfamily,
  basicstyle=\linespread{0.9}\ttfamily,
	breaklines=true,
  xleftmargin=\parindent,
  belowskip=\bigskipamount,
  aboveskip=\bigskipamount,
  tabsize=4,
  morecomment=[l][\color{greencomments}]{///},
  morecomment=[l][\color{greencomments}]{//},
  morecomment=[s][\color{greencomments}]{{(*}{*)}},
  morestring=[b]",
  showstringspaces=false,
  literate={`}{\`}1,
  frame=none,
  showlines=false,
  %frame=single,
  stringstyle=\color{redstrings},
}



\newcommand{\todo}[1]{{\noindent\small\color{red} \framebox{\parbox{\dimexpr\linewidth-2\fboxsep-2\fboxrule}{\textbf{TODO:} #1}}}}

\newcommand*{\CHAPTERSROOT}{../.}	% root path for chapters.
\newcommand*{\chapterprefix}{02}	% your chapter number.

\newcommand{\mkwd}[1]{\ensuremath{\mathsf{#1}}}
\usepackage{subfig}
\usepackage{tikz}
\usetikzlibrary{matrix,fit}
\usepackage[scaled]{beramono}
\usepackage[T1]{fontenc}
\usepackage{listings}
\lstset{
  basicstyle=\ttfamily
}

\newcommand{\mypar}[1]{\vspace{1em}\textbf{#1}\quad}

\newcommand{\calcwd}[1]{\textbf{\textsf{#1}}}
\newcommand{\actsend}[2]{\calcwd{send} :\ #1 \: #2}
\newcommand{\pid}[1]{\mkwd{ActorRef}(#1)}
\newcommand{\seq}[1]{\overrightarrow{#1}}
\newcommand{\totheleft}[1]{\begin{flushleft}#1\end{flushleft}}

\newcommand{\chan}[1]{\mkwd{Chan}(#1)}
\newcommand{\var}[1]{\mathit{#1}}
\newcommand{\newch}{\calcwd{newCh}}
\newcommand{\gvsend}[2]{\calcwd{send} \: #1 \: #2}
\newcommand{\gvrecv}[1]{\calcwd{receive} \: #1}
\newcommand{\gvclose}[1]{\calcwd{close} \: #1}
\newcommand{\letintwo}[2]{\calcwd{let} \: #1 = #2 \: \calcwd{in}}
\newcommand{\bl}{\begin{array}{l}}
\newcommand{\blt}{\begin{array}[t]{l}}
\newcommand{\el}{\end{array}}
\newcommand{\defeq}{\triangleq}
\newcommand{\metadef}{\mkwd}
\newcommand{\lto}{\multimap}
\newcommand{\one}{\mathbf{1}}
\newcommand{\gvdual}[1]{\overline{#1}}
\newcommand{\gvend}{\mkwd{End}}
\newcommand{\gvout}[2]{{!}#1{.}#2}
\newcommand{\gvin}[2]{{?}#1{.}#2}
