%!TEX root = ../../editor.tex

%%%%%%%%%%%%%%%%%%%%%%%%%%%%%%%%%%%%%%%%%%%%%%%%%%%%%%%
%
% chapter03 - Execution Models and Debugging
%
%%%%%%%%%%%%%%%%%%%%%%%%%%%%%%%%%%%%%%%%%%%%%%%%%%%%%%%

%
% >>>>>>>>>>>>>>> PLEASE NOTE <<<<<<<<<<<<<<<
%
% This file is not stand-alone compileable as it is, to make it compileable while writing uncomment the preamble below.
% In this case, you also have to uncomment the begin/end document statements.
% You can outcomment the preamble and the begin/end document statements again or erase them when handing in your contribution.
%
% If you use BibTex for your bibliography, please use \putbib[bibliography] to print your reference (see end of this file).
%
% you can use paths relative to your chapter dir, e.g. \figure{assets/fig1}.
%
% >>>>>>>>>>>>>>>>>>>><<<<<<<<<<<<<<<<<<<<<<<

%%%%%%%%%%%%%%%%%%%%%%%%%%%%%%%%%%%%%%%%%%%%%%%%%%
%% you can uncomment the following preamble during development to make this file compileable.
%% Note that you need the svmult.cs file inside your chapter root dir to make this work.
%% Also note that if you need additional packages etc., you can add them here, but please
%% mark them somehow so the editor of this book knows you need them in the final book.
%% When you hand in your contribution, please uncomment or remove the preamble again.
%%%%%%%%%%%%%%%%%%%%%%%%%%%%%%%%%%%%%%%%%%%%%%%%%%
%%%%%%%%%%%%%%%%%%%%%%%%%%%%%%%%%%%%%%%%%%%%%%%%%%% start of preamble
%\documentclass[
%graybox,
%envcountchap,
%%natbib
%]{svmult}
%
%\usepackage[utf8]{inputenc}
%%\usepackage{type1cm}        % activate if the above 3 fonts are 
%% not available on your system
%
%\usepackage{makeidx}         % allows index generation
%\usepackage{graphicx}        % standard LaTeX graphics tool
%% when including figure files
%\usepackage{multicol}        % used for the two-column index
%\usepackage[bottom]{footmisc}% places footnotes at page bottom
%
%\usepackage{newtxtext}       % 
%\usepackage{newtxmath}       % selects Times Roman as basic font
%
%% \usepackage{natbib}
%\usepackage{footmisc}
%
%%% Additional packages added. Add necessary packages here.
%%\usepackage[english]{babel}
%\usepackage{siunitx}
%\usepackage{amssymb}
%\usepackage{pifont}
%\usepackage{xcolor}
%\usepackage{tabularx}
%\usepackage{listings}
%\usepackage{booktabs}
%\usepackage{hyperref}
%\usepackage{url}
%\usepackage{mathtools}
%\usepackage{lipsum}
%\usepackage{import}
%\usepackage{bibunits}
%\usepackage{acronym}
%\usepackage[nottoc]{tocbibind}
%\usepackage{numberpt}
%
%\newcommand*{\CHAPTERSROOT}{../.}	% root path for chapters.
%\newcommand*{\chapterprefix}{03}	% your chapter number.
%
%\makeindex % used for the subject index
%%%%%%%%%%%%%%%%%%%%%%%%%%%%%%%%%%%%%%%%%%%%%% end of preamble

%% uncomment the \begin{document} statement to make this file stand-alone compileable.
%\begin{document}

\begin{bibunit}
	
	\title*{Execution Models and Debugging}
	\author{Stefan Marr and Hidehiko Masuhara}
	
	\institute{
		Stefan Marr \at University of Kent, United Kingdom, \email{s.marr@kent.ac.uk}
		\and Hidehiko Masuhara \at Tokyo Institute of Technology, Japan, \email{masuhara@is.titech.ac.jp}
		\and Carlos Varela \at Rensselaer Polytechnic Institute, USA, \email{cvarela@cs.rpi.edu}
		\and Masahiro Yasugi \at Kyushu Institute of Technology, Japan, \email{yasugi@ai.kyutech.ac.jp}
		\and Elisa Gonzalez Boix \at Vrije Universiteit Brussel, Belgium, \email{egonzale@vub.ac.be}
		\and Torsten Grust \at University of Tübingen, Germany, \email{torsten.grust@uni-tuebingen.de}
	}
	\maketitle
	
	\abstract{Please place your abstract here.}
	
\begin{note}
Please include at least 2 illustrations!
\end{note}

\section{Introduction}

\begin{note}
[up to 2 pages]
What the chapter is about. Define the subject, e.g., what are CPS?
How is relevant to distributed systems and programming languages?
Promise of what development in PL4DS could achieve.
\end{note}

\section{State of the Art}

\begin{note}
[up to 10 pages]
How is <subject> being used? What are motivating applications? How easy is it to program these correctly and efficiently?
What are recent developments?
\end{note}

\subsection{Background}
\begin{note}
What are the current notations? Theoretical foundations?
\end{note}

\section{Research Directions}
\begin{note}
[at least 5 pages]
What are current weaknesses and open problems? How can they be addressed? Relate back to applications.
What are promising research directions? How can they address the needs of applications?
What are the additional capabilities that could be enabled? Which novel applications will become possible?
\end{note}
  
\section{Outlook and Conclusion}
\begin{note}
[up to 3 pages]
Why is it now appropriate to pursue this work?
<evangelical section>
summarize promise
\end{note}

\section*{Notes from Shonan}
\sm{test}
\begin{noteco}
  Execution Models

  Topics
  •	GPUs
  •	Virtual Machines
  •	Containers
  •	PaaS, Function-as-a-service, Serverless computing
  •	Database systems
  •	Programming Model, language abstractions
     o	Language feature support in runtimes
       - Tail calls
       - Coroutines
       - Laziness
       - Back-tracking
  •	Communication model
     o	Message passing, MPI
     o	PGAS
  •	Dynamic updating
  •	Version management
  •	Dynamic scaling, elasticity
  •	Edge computing
     o	asynchronous 
     o	Partial views
     o	Real-time constraints
  •	(Blockchain)
     o	Dynamically updated consensus protocol (e.g. Tezos)
       - Agreed by consensus
     o	Execution limited by available “gas”
       - Limits the number of instructions that can be executed
       - Security concerns, turing completeness
  •	Runtime services for
     o	Monitoring
     o	Tracing
     o	Replay
     o	As fundamental enabler for
       - (deterministic) re-execution
       - fault tolerance
       - fault mitigation
       - Debugging
     o	Enforcing idempotent design
     o	Fault tolerance in heterogeneous language environments
       - CORBA had exception support
       - MPI has now support for fault tolerance
       - Distributed Transactions
  •	On the level of operations
  •	Research topic in the 90ies
  •	Heterogeneity (languages/systems) not worked out
       - Distributed concurrency/locking support (very similar problem)
       - Problem: these things are typically external
       - We are used to high-level languages hiding most of these things
       - How can we get to that state even across languages/runtimes?
  •	Security
     o	Enclaves, completely encrypted, not leaking any data
     o	Limitations on system calls
     o	Memory pages leaving enclave are encrypted
     o	Code is authenticated/authorized/signed
     o	Code knows that it runs on a genuine CPU, not some virtual machine
     o	Data privacy
     o	Classic issues with distributed systems
     o	Safety
     o	Multi-tenant systems
  •	Streaming, reactivity
     o	Enforcing synchronicity for correctness
     o	Challenge with multiple sources, and different data rates or clocks (glitches are major issue)
  •	Standard libraries
  •	Application Scenarios
     o	Microservices, distributed services
     o	Machine learning, tensor flow
     o	Runtime environments may need to be tailored to specific needs
       - For scale, for performance, for fault tolerance

  Open Questions, Challenges
  •	Fault tolerance
  •	Making Function as a Service efficient
  •	Tension between Computing and Data resources
     o	Moving data to function or function to the data
     o	In cloud, edge, mist, fog computing environments
     o	Fault tolerance, real time constraints
  •	In distributed systems, inconsistency is the norm
     o	How much consistency can we tolerate?

  General Notes, Discussions
  •	What do execution platforms need to provide?
     o	One discussion about actors isolation
     o	Actor isolation prevents shared memory
     o	Is a nice engineering property, but has possible performance tradeoffs 
  •	Can we automatically decide what the most efficient execution platform is?


  Important Questions
  •	Cost-effective/performance function-as-a-service, while preserving deployment/infrastructure benefits, security/isolation properties
  •	Security
  •	Fault tolerance for distributed/cloud 
  •	Support for interoperability of components, where languages support different features
\end{noteco}  
  
	\section*{Appendix}\label{appendix}
	
	Please place your appendix content here, if applicable.
	
	%%%%%%%%%%%%%%%%%%%%%%%%%%%%%%%%%%%%%%%%%%%%%%%%%%%%%%%%%%%%%%%%%%%%%%%%%%%%%%%%%%%%%%%%%%%%%%%%%%%%%%%%
	%% For your bibliography, you should use a bibtex .bib file and include it here.
	%% Note that the final reference lists styling might differ because it'll be styled in unified book layout.
	
	% \biblstarthook{
	%	text inserted here will be printed before the actual list of references, but only if there is at least one reference to %display. Delete this section if you don't need it.
	%}
	
	% \nocite{*}		%% uncomment if uncited references should be listed in the bibliography.
	
	%% uncomment and state path to your .bib to use a bibtex file as your bibliography.
	%% NOTE: relative paths don't work in \putbib => During development, you might delete the "\CHAPTERSROOT/chapter\chapterprefix/" part to refer to your bib file. When you're done, please make this path absolute by adding the prefix again.
	%%
	% \putbib[\CHAPTERSROOT/chapter\chapterprefix/bibliography] %
	
\end{bibunit}
	
%% uncomment the \end{document} statement to make this file stand-alone compileable.
%\end{document}